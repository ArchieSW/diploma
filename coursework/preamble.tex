\usepackage{geometry}
\usepackage[T1]{fontenc}
\usepackage[utf8]{inputenc}
\usepackage[english,russian]{babel}
\usepackage{amsmath}
\usepackage{amsthm}
\usepackage{amssymb}
\usepackage{fancyhdr}
\usepackage{setspace}
\usepackage{graphicx}
\usepackage{colortbl}
\usepackage{tikz}
\usepackage{pgf}
\usepackage{subcaption}
\usepackage{listings}
\usepackage{indentfirst}
\usepackage[colorlinks,citecolor=blue,linkcolor=blue,bookmarks=false,hypertexnames=true, urlcolor=blue]{hyperref}
\usepackage{indentfirst}
\usepackage{mathtools}
\usepackage{booktabs}
\usepackage[flushleft]{threeparttable}
\usepackage{tablefootnote}

\usepackage{chngcntr} % нумерация графиков и таблиц по секциям
\counterwithin{table}{section}
\counterwithin{figure}{section}

\graphicspath{{../graphics/}}%путь к рисункам

\makeatletter
\renewcommand{\@biblabel}[1]{#1.} % Заменяем библиографию с квадратных скобок на точку:
\makeatother

\geometry{left=2.5cm}% левое поле
\geometry{right=1.5cm}% правое поле
\geometry{top=1.5cm}% верхнее поле
\geometry{bottom=1.5cm}% нижнее поле
\renewcommand{\baselinestretch}{1.5} % междустрочный интервал


\newcommand{\bibref}[3]{\hyperlink{#1}{#2 (#3)}} % biblabel, authors, year
\addto\captionsrussian{\def\refname{Список литературы (или источников)}}

\renewcommand{\theenumi}{\arabic{enumi}}% Меняем везде перечисления на цифра.цифра
\renewcommand{\labelenumi}{\arabic{enumi}}% Меняем везде перечисления на цифра.цифра
\renewcommand{\theenumii}{.\arabic{enumii}}% Меняем везде перечисления на цифра.цифра
\renewcommand{\labelenumii}{\arabic{enumi}.\arabic{enumii}.}% Меняем везде перечисления на цифра.цифра
\renewcommand{\theenumiii}{.\arabic{enumiii}}% Меняем везде перечисления на цифра.цифра
\renewcommand{\labelenumiii}{\arabic{enumi}.\arabic{enumii}.\arabic{enumiii}.}% Меняем везде перечисления на цифра.цифра


% Команда для титульного листа
\newcommand{\maketitlekrvkr}[9]{%
% 1: тип работы (КР или ВКР)
% 2: ФИО автора
% 3: тема работы
% 4: направление подготовки
% 5: образовательная программа
% 6: факультет
% 7: рецензент (оставить {} если нет)
% 8: руководитель (оставить {} если нет)
% 9: город и год (например: Нижний Новгород, 2025)

\begin{center}
\setstretch{1.2}
ФЕДЕРАЛЬНОЕ ГОСУДАРСТВЕННОЕ\\
АВТОНОМНОЕ ОБРАЗОВАТЕЛЬНОЕ УЧРЕЖДЕНИЕ\\
ВЫСШЕГО ОБРАЗОВАНИЯ\\
«НАЦИОНАЛЬНЫЙ ИССЛЕДОВАТЕЛЬСКИЙ УНИВЕРСИТЕТ\\
«ВЫСШАЯ ШКОЛА ЭКОНОМИКИ»\\[1ex]
\textit{#6}\\[4ex] % Факультет

\textit{#2}\\[3ex] % ФИО автора курсивом

\textbf{#3}\\[2ex] % Название темы

\ifthenelse{\equal{#1}{КР}}{%
Курсовая работа%
}{%
Выпускная квалификационная работа -- БАКАЛАВРСКАЯ РАБОТА%
}\\
по направлению подготовки #4\\
образовательная программа\\
«#5»\\[6ex]

% Нижняя часть с рецензентом и руководителем
\noindent
\begin{minipage}[t]{0.49\textwidth}
\raggedright
\ifthenelse{\equal{#7}{}}{}{%
Рецензент\\
#7\\[6ex]
}
\end{minipage}
\hfill
\begin{minipage}[t]{0.49\textwidth}
\raggedleft
\ifthenelse{\equal{#8}{}}{}{%
Руководитель\\
#8\\[6ex]
}
\end{minipage}

\vfill
#9
\end{center}
}
